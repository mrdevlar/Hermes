\vspace*{40px}
\begin{displayquote}
"If people do not believe that mathematics is simple, it is only because they do not realize how complicated life is." 
\end{displayquote}

\begin{flushright}
\textit{John von Neumann}
\end{flushright}

\vspace*{25px}

\section*{Acknowledgments}

No man is an island.

While a single name appears on the cover of this document, it should not be implied that anything was, or could be done, alone. Everyone is the product of their environments and I've had the esteemed privilege to be assisted and inspired by many wonderful people. 

So my great appreciation to Ewa Podkolinska, Jerzy Podkolinski, Alina Kroeker, Raymond Kroeker, Tom Ruette, Wannes Meert, Mathias Verbeke, Gregory Schiano, Jan Aerts, Gerda Claeskens, Luc De Raedt, Roel Braekers, Geert Verbeke, Eric Schmitt, Rui Barbosa, Stijn Van Weezel, Akshat Dwivedi, Irzam Hardiansyah, Zhu Meng, Sytze Elzinga, Justin Fischedick and most importantly Rianne Hartemink.

Thank you for your support and motivation. 

\begin{flushright}
\textit{Richard Podkolinski}
\\
2016 
\\
Den Haag
\\
The Netherlands
\end{flushright}

\vspace*{25px}

\section*{Intended Audience}

Good research is the kind that is capable of taking a complex topic making it accessible to a broader audience. In an effort to adhere to that standard, the intended audience of this thesis is not a pure mathematician, rather a general practitioner in the area of predictive maintenance or reliability analysis. Emphasized is the development of understanding and intuition for the underlying models and how to handle edge cases. 

While great effort is undertaken to be as accessible as possible, a certain degree of mathematical development is required to understand this text. Recommended is an understanding of the calculus of probabilities at the level of Blitzstein and Huang\cite{Blitzstein2014} as well as a first course on Bayesian linear models at the level of McElreath\cite{McElreath2016}. A basic understanding of how basic linear models are fit and their performance evaluated is required.

To ensure access to rigorous results, an extensive list of references and resources are made available at the end of each chapter, dubbed "Extended Reading". These resources elucidate many of the technical aspects of the modeling process and mathematical foundations which are beyond the scope of this text.

\newpage

\section*{Abstract}


The evolution of predictive maintenance strategies is not universally applicable in all domains. The increased ability of organizations to capture a wider breadth and greater volume of data has lead to a narrow research focus that cannot be easily translated to all domains. Industrial-scale solar power is one of these domains, which has limited access to condition monitoring data. Yet, the need in the domain for predictive maintenance is greater than ever.

This thesis addresses the development of a predictive maintenance process in informationally sparse systems, specifically using the photovoltaic inverters as the object of interest. It seeks to fill the void in the current literature on predictive maintenance in environments where there is limited access to operational data. It addresses the problem through the use of Bayesian time-to-event analysis, a statistical modeling approach used to estimate the remaining lifetime of an object. As such, it progresses through the process of developing a predictive maintenance system starting from the mathematics of time-to-event analysis through to the practical concerns of deploying an process. Finally, it presents an application using simulated data using Stan, the probabilistic programming language.


\newpage

\section*{List of Abbreviations}

\begin{itemize}

\item[] PV -  Photovoltaic
\item[] BOS - Balance of Systems
\item[] AC - Alternating Current
\item[] DC - Direct Current
\item[] kWh - KiloWatt Hours

\item[] pdf - Probability Density Function
\item[] cdf - Cumulative Distribution Function
\item[] MLE - Maximum Likelihood Estimation
\item[] MCMC - Markov Chain Monte Carlo
\item[] HMC - Hamiltonian Monte Carlo

\item[] C-Index - Concordance Index
\item[] WAIC - Widely Applicable Information Criterion
\item[] DIC - Deviance Information Criterion

\item[] MSE - Mean-Squared Error
\item[] MAE - Mean-Absolute Error

\item[] AUC - Area Under Curve
\item[] ROC - Receiver Operating Conditions


\end{itemize}

\newpage

\section*{List of Symbols}

\begin{itemize}

\item[] $T$ - Random variable for a lifetime
\item[] $t$ - Realization of the above random variable; an instance of time.
\item[] $Y$ - Random variable for observed lifetimes
\item[] $y$ - Realization of the above random variable; an observed instance of time.
\item[] $d$ - Censoring Indicator
\item[] $\theta$ - Vector of Unknown Parameters
\item[] $\theta^s$ - Vector of Parameters from Completed Simulations
\item[] $\Gamma(\cdot)$ - The Gamma Function
\item[] $f(\cdot)$ - Probability Density Function
\item[] $F(\cdot)$ - Cumulative Distribution Function
\item[] $S(\cdot)$ - Survival Function (Reliability Function)
\item[] $h(\cdot)$ - Hazard Function (Conditional Failure Rate, Intensity, Force of Mortality)
\item[] $H(\cdot)$ - Cumulative Hazard Function
\item[] $h_0(\cdot)$ - Baseline Hazard Function
\item[] $H_0(\cdot)$ - Baseline Cumulative Hazard Function
\item[] $I(\cdot)$ - Indicator Function
\item[] $|\cdot|$ - Cardinality of a Set

\item[] $\mathbb{P}_c$ - Probability of Concordance
\item[] $\mathbb{P}_d$ - Probability of Discordance

\item[] $\mathcal{G}$ - A Graph
\item[] $\mathcal{V}$ - Set of Graph Vertices 
\item[] $\mathcal{E}$ - Set of Graph Edges 

\item[] $\text{elpd}$ - Expected Log Pointwise Predictive Density for a New Dataset
\item[] $\text{lpd}$ - Log Pointwise Predictive Density
\item[] $\hat{p}$ - Effective Number of Parameters

\end{itemize}

