\chapter*{Introduction}

Maintenance is not sexy, but it is essential. It is the immune system of the modern industrialized world. Without it, the myriad of systems we depend on for safeguarding our collectively quality of life would rapidly deteriorate and disappear. Yet, few sing the praises of the multitude of technicians and engineers which are responsible for ensuring trains run on time, computer systems remain accessible or that bridges do not suddenly collapse. Paradoxically, this lack of visibility is a testament to the uninterrupted success of these activities. 

The preceding decades have brought with them rapid innovation in the area of maintenance. What has historically been a manual and labor-intensive activity is increasingly becoming a scientific process, one that leverages technology to optimize operations. Modernization has meant going beyond traditional methods of maintenance based on reactions to failure or scheduled inspection and moving towards those that utilize data and predict failure. Such a trend can have a widespread impact on operational expenses. Reactive methods generally incur greater costs due to downtime and a need for more expensive replacement parts. Scheduled inspections share similar pitfalls, requiring downtime, specialized machinery and also running the risk of inadvertent damage to the objects they seek to maintain. Predictive maintenance minimizes many of these drawbacks. Judgments about the state of a device are made as a result of continuous monitoring of actual operating conditions\cite{Mobley2002}. This reduces downtime and allows for a better allocation of maintenance resources. Most importantly, predictive maintenance enables faults to become apparent *before* failure allowing for contingencies, such as repair or replacement, to ensure constant output.


The transition toward predictive maintenance owes much to the continued development of sensor technology. More reliable, accurate, and most importantly, inexpensive sensors have enabled continuous monitoring in a wider range of domains. Many modern devices come with built-in sensors that record output or other context-specific data that can be used to assess health. Increasingly, many also include connectivity that allows telemetry to be retrieved remotely on demand. This reality is likely what has spurred widespread innovation in the field, with more than half of the publications on predictive maintenance emerging within the last decade.\footnote{Based on Search: Limo = 703 / 989 and Google Scholar = 3300 / 5450} Unfortunately, this data gluttony has also led to a form of tunnel vision, with the majority of this work dedicated to enabling predictive maintenance exclusively in informationally rich contexts. As a result, authors are predominantly preoccupied with domains where data is high dimensional, time-dependent, with fine granularity and encoded without error. 


While the future may make these ideal circumstances ubiquitous, there is still a wide set of domains where this is not a reality. Sensors may have become increasingly cheap and plentiful but they can still be expensive in absolute terms. The majority of the sensors included in new products are meant to extend functionality and the continuous monitoring they enable is an fortunate but inadvertent consequence. Yet, for many devices the inclusion of additional sensors means an increase in manufacturing costs, which is often prohibitive. No product developer of sound mind would argue for the inclusion of a sensors in a t-shirt to prevent stitching failure. The sensors would vastly increase the final cost of the original product. However, this kind of cost-benefit dilemma is found throughout numerous domains where the sensor costs are higher than the manufacturing cost of the product. In these domains, data is rarely high dimensional, time-dependent, fine and encoded without error. Despite these limitations, there is still a need for predictive maintenance. 

This thesis will address the development of a predictive maintenance system in informationally sparse domains. In so doing, it will attempt to fill the void in the current literature on predictive maintenance in domains where limited access to operational data is the norm. It will attack the problem from the perspective of Bayesian time-to-event analysis, a statistical modeling approach which is used to estimate the remaining lifetime of an object. It will then provide an application using simulated data and Stan, the probabilistic programming language. The following chapters encompass the development of this strategy. 

The first chapter examines predictive maintenance and introduces an example domain meant to provide a business case for the further analysis. The solar industry with its heavy reliance on dwindling government subsidies and firm need to reduce costs prove an ideal domain for such systems. Further, the photovoltaic inverter, a device with limited output found at every solar power plant is a highly relevant object of study. 

The second chapter considers methodology, providing the probabilistic and statistical foundations required to understand Bayesian time-to-event models. These foundations are then used to introduce more complex models centered around the Multiplicative Hazard Model with shared Frailties. Finally, estimation procedures for these models using Bayesian Methods are introduced with a brief introduction to Hamiltonian Monte Carlo. 

The third chapter covers the analytical process using simulated photovoltaic inverter data. It restates the goal of the analysis in concrete terms, then examines the steps needed to develop a predictive maintenance process. Feature engineering is examined, including what resources must be made available for such a system to be put into place. Then, data management and the simulation process are discussed. Stan, a recently development in fitting Bayesian models is examined and its implementation method is described. Then, an exploration of how to assess the predictive performance in the context of time-to-event models is provided. It concludes with a step-by-step construction of the time-to-event models in Stan. 

Finally, we conclude with a summary of the material and provide a general guidelines for time-to-event modeling in a predictive context.




